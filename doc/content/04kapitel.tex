%!TEX root = ../dokumentation.tex
\chapter{Datenbank}

\section{Hosting}
Zum Hosten der Datenbank wird die \href{https://aws.amazon.com/de/rds/mariadb/}{Amazon RDS for MariaDB} mit der MariaDB Version 10.2.21 verwendet. \\
\glqq\href{https://aws.amazon.com/de/rds/}{Amazon Relational Database Service (Amazon RDS)} ist ein Webservice, der das Einrichten, Betreiben und Skalieren einer relationalen Datenbank in der AWS Cloud vereinfacht.\grqq  Amazon RDS unterstützt bis zu sechs verschiedene Datenbank-Engines, z.B. MySQL, PostgreSQL oder auch MariaDB. \cite{AWSRDS}

Der Vorteil einer Cloud-basierten Datenbank ist, dass jeder Entwickler problemlos darauf zugreifen kann.\\
Um auf diese Datenbank zugreifen zu können, unterstützt AWS sogenannte Security Groups. Diese Gruppen dienen als virtuelle Firewall. Man teilt sie einer Ressource wie z.B. der Datenbank oder dem Elastic Beanstalk Server zu. In einer Security Group wird spezifiziert, welche IPs und sonstigen Ressourcen Zugriff haben.\\
Da während der Entwicklungszeit die IP Adresse eines Entwicklers sich täglich geändert hat, wurde Zugriff von Überall erlaubt. Dies stellt natürlich ein erhöhtes Sicherheitsrisiko dar, dennoch braucht man Anmeldedaten für die Datenbank. Nach der Entwicklungsphase wurde dieser Zugang entfernt.

\section{Verwendung von \$\_SERVER}
Damit die Anmeldedaten für die Datenbank nicht blank im Quellcode zu finden sind, wurden diese in dem \$\_SERVER Array des Webservers gespeichert. Dieses Verfahren wurde gewählt, da ein öffentliches Github Repository erstellt wurde und somit sonst für jeden sichtbar wäre.

Um die Daten auf dem lokalen Xampp Server verwenden zu können, werden diese in der \lstinline{httpd-xampp.conf} gespeichert. Im folgenden Codeblock werden die Anmeldedaten im Modul \lstinline{env_module} definiert.
\newpage
\begin{lstlisting}[caption={httpd-xampp.conf Umgebungsvariablen}, language=bash]
#
# XAMPP settings
#

<IfModule env_module>
    ...
    SetEnv RDS_DB_NAME "planning_poker"
    SetEnv RDS_HOSTNAME "planning-poker.cztpemauxamy.eu-central-1.rds.amazonaws.com"
    SetEnv RDS_PASSWORD "<password>"
    SetEnv RDS_PORT "3306"
    SetEnv RDS_USERNAME "admin"
</IfModule>
\end{lstlisting}

Nun kann man innerhalb des Programmcodes auf das \$\_SERVER Array zugreifen um die zuvor definierten Werte aufzurufen. Dieser Codeblock, aus der Datei Model/ModelBase.php, zeigt die Funktion zur Erstellung der Datenbankverbindung. 

\begin{lstlisting}[caption={Funktion getPdo()}, language=PHP]
/**
* GetPDO: returns a PDO Connection
* @author Luca Stanger
* @return Database
*/
public function getPdo()
{
	if (self::$_db === null) {
		self::$_db = new \PlanningPoker\Model\Database(
			$_SERVER['RDS_HOSTNAME'],
			$_SERVER['RDS_PORT'],
			$_SERVER['RDS_DB_NAME'],
			$_SERVER['RDS_USERNAME'],
			$_SERVER['RDS_PASSWORD'],
			'utf8'
		);
	}

	return self::$_db;
}
\end{lstlisting}

Damit der AWS Elastic Beanstalk Server ebenfalls auf die Daten über \$\_SERVER zugreifen kann, werden diese unter den Konfigurationseinstellung in AWS als Umgebungseigenschaften definiert.

\section{Schema}
In der Abbildung \ref{datenbank} ist das Schema der Datenbank zu sehen.

\begin{figure}[H]
	\centering
  \includegraphics[width=0.74\textwidth]{images/database.png}
	\caption{Visualisierung der Datenbank}
	\label{datenbank}
\end{figure}

Zu sehen sind die einzelnen Tabellen der Datenbank mit Verknüpfungen zu anderen Tabellen. In einer Tabelle sind die einzelnen Spalten beschrieben. Primärschlüssel haben einen goldenen Schlüssel, Fremdschlüssel eine rote Raute und normale Spalten eine blaue Raute. Außerdem werden die Typen der Spalten gezeigt. Alle Verbindungen in diesem Schema beschreiben eine 1:N Beziehung mit ON DELETE CASCADE.

Der Vorteil dieses Schemas ist, dass keine überflüssigen Daten vorhanden bleiben. Wenn z.B. ein Benutzer sich löscht, werden automatisch alle Daten dieses Benutzers aus den verbundenen Tabellen gelöscht. Die Reviews, Votes, Participations und erstellten Lobbies des Benutzers werden gelöscht. Dadurch, dass eine Lobby gelöscht wird, werden die Issues dieser Lobby, die Votes und Participations gelöscht.