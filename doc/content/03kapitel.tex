%!TEX root = ../dokumentation.tex
\chapter{Database}\label{ch:db}
Da das Projekt auf dem MEAN-Stack basieren solle, wurde die NoSQL Datenbank MongoDB verwendet.\\
Der erste Schritt war es, eine MongoDB mithilfe von \hyperlink{https://www.mongodb.com/cloud/atlas}{MongoDB Atlas} aufzusetzen, denn man wollte ein funktionierendes Backend als erstes haben damit das Frontend ebenfalls bereits anfangen konnte. Als Student erhält man ein Budget von 300\$ und man kann außerdem kostenlose Datenbanken erstellen.\\
Der nächste Schritt war es, die Datenbank als Container zu erstellen. Da bereits die Verwendung einer Docker Compose in Ansicht war, wurde die Datenbank direkt in der Compose Datei erstellt. Diese kann mit dem Befehl docker-compose up --build database gestartet werden.

\section{Zugriff}
Mongo Atlas
mit docker compose up --build database
warum out admin dran ist
datagrip

\section{Verwendung}