%!TEX root = ../dokumentation.tex
\chapter{Database}\label{ch:db}
Da das Projekt auf dem MEAN-Stack basieren solle, wurde die NoSQL Datenbank MongoDB verwendet.\\
Der erste Schritt war es, eine MongoDB mithilfe von \hyperlink{https://www.mongodb.com/cloud/atlas}{MongoDB Atlas} aufzusetzen. Der Grund dafür war, da bereits Teile des Backends entwickelt waren und eine einfache Datenbank gebraucht wurde zum testen. Als Student erhält man ein Budget von 300\$ und man kann außerdem kostenlose Datenbanken erstellen.\\
Der nächste Schritt war es, die Datenbank als Container zu erstellen. Da bereits die Verwendung einer Docker Compose in Ansicht war, wurde die Datenbank direkt in der Compose Datei erstellt. Diese kann mit dem Befehl docker-compose up --build database gestartet werden.